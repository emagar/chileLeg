% Created 2019-03-04 Mon 10:17
% Intended LaTeX compiler: pdflatex
\documentclass[article,letterpaper,times,12pt,listings-bw,microtype]{article}
\usepackage[utf8]{inputenc}
\usepackage[T1]{fontenc}
\usepackage{graphicx}
\usepackage{grffile}
\usepackage{longtable}
\usepackage{wrapfig}
\usepackage{rotating}
\usepackage[normalem]{ulem}
\usepackage{amsmath}
\usepackage{textcomp}
\usepackage{amssymb}
\usepackage{capt-of}
\usepackage{hyperref}
\usepackage[letterpaper,right=1in,left=1in,top=1in,bottom=1in]{geometry}
\usepackage{xr}
\externaldocument[app-]{urge14appendix}
\author{Double blind review}
\date{\today}
\title{Resubmission of "Presidents on the Fast Track: Fighting Floor Amendments with Restrictive Rules" to the \emph{Journal of Politics} (Ms. No. 182014)}
\hypersetup{
 pdfauthor={Double blind review},
 pdftitle={Resubmission of ``Presidents on the Fast Track: Fighting Floor Amendments with Restrictive Rules'' to the \emph{Journal of Politics} (Ms. No. 182014)},
 pdfkeywords={},
 pdfsubject={},
 pdfcreator={Emacs 24.5.1 (Org mode 9.1.7)}, 
 pdflang={English}}
\begin{document}

\maketitle
%\tableofcontents

%\newpage

\noindent
We are grateful for the opportunity to revise and resubmit our manuscript. In this document, we paraphrase and respond to very helpful critiques and comments provided by the editorial team and reviewers. We are confident that the manuscript has been greatly improved by the review process, and we are indebted to each one of you for your thoughtful remarks. Thank you very much for your time, kind words, and professional service. 

\section{Responses to Reviewer \#1:}
\subsection {Are there other tools available for presidents to generate decisions faster? What happens when the president has decree powers? Are the odds that the president will qualify a bill as urgent the same when he has decree power?} 

This is an important question that we did not adequately consider in the version we submitted for review. Reviewer \#1 suggests ``to make clear that consideration of this condition will not impact either the explanatory power of the model or the results of the analysis.'' In this new version of the manuscript we were careful  to differentiate among the ways in which the differet tools that enhance the role of presidents in the legislative process in Latin America interact (page 3). We made explicit the position that urgency authority occupies among the different presidential tools for unorthodox lawmaking.

In particular, this version elaborates a more nuanced discussion of executive decree authority. From a theoretical perspective, we assume that there is a choice to enact legislation using decrees or congressional statute, and that under specific conditions presidents simply cannot enact decrees (we dropped the cite to keep anonymity). In those cases presidents are forced to see legislation through the congressional route, and it is under such circumstances that presidents may put this other tool to use. 

It was great that Reviewer \#1 pushed us to question if for cases that share  both institutions (Brazil, Chile, Colombia), decrees might in fact reduce urgency attractiveness. We surveyed the literature and thought deeply about this issue. Figueiredo and Limongi report that Brazilian presidents rarely declare bills urgent, and we include a quote of them saying that there is a natural preference for more versatile decrees. This ``natural preference'' is reversed in Chile, where only two decrees have been enacted since 1990, but instead, urgencies are used very frequently. We argue that even where presidents can issue decrees constitutionally, they often do not, and not for lack of haste (urgency). Pereira, Power, and Rennó (2005, 2008) find that Brazilian presidents are not systematically more likely to rely on decrees under economic pressure, for instance. Thus, the main point we highlight is that urgency authority is a tool that is pulled out of the toolbox only after the decree route has been discarded. Once a bill enters the standard congressional path, urgency authority may be used, and the manuscript tries to explain why its use occurs when it does. In Section 2 (Urgency Decree Authority) and Section 5 (Discussion) we include an extended discussion of the insights from scholars analyzing the tools presidents have in Latin America, and how the effect and use of decrees is different from that of urgency authority. 



%Reviewer: Different from the US president, some Latin American presidents count on constitutional decree powers, which allow them to issue decrees having the force of law without prior authorization of the Congress. Usually limited to specific policy areas, this decree power allows the president to pursue his preferences and overturn legislative resistance to them.   As is known, this kind of provision grants the president first-move advantages to change the "status quo" and implement a decision close to his own preferences. This forces legislators to respond to the president's decision instead of the previous status quo. In addition, as emergency decrees, they contain provisions that make introducing floor amendments more restrictive and, consequently, expedite the legislative process. Therefore, are the odds that the president will qualify a bill as urgent the same when he has decree power? I suggest the author discuss this point, even if only to make clear that consideration of this condition will not impact either the explanatory power of the model or the results of the analysis. Empirical analysis is done only for the Chilean case where, as in the US, constitutional decree power is not granted. However, it is important to consider this point, since this manuscript seeks to offer a model for comparative analysis.
 

\subsection{How is urgency authority used in different policy areas?}
Reviewer \#1 inquired about the effect of other policy areas on the use of urgency authority, given that some areas are more important than others and therefore, more strategic for the president. This is a significant point: different issues carry different stakes. 
In our previous version of the manuscript we had controlled for Finance Comittee referral because this is a simple but powerful measure of relevance. It is a constitutional requirement that every bill authorizing spending be referred to and reported by the Finance (\emph{Hacienda}) Committee. The committee, in tandem with the Finance Ministry, decides whether or not to appropriate the funds. Our claim was straightforward: while there must be exceptions, bills not authorizing spending are on some dimension less important politically than bills authorizing spending. Doubtless, this leaves other issues unattended, several possibly of great relevance. So we decided to extend the work we had done to tackle this factor associated to the variety of issues that may impact the likelihood of reliance on urgency authority.

Admittedly, controlling for issue/policy area effects is a difficult task, but we carried it out as best as possible given our constraints. We classified bills into broad issue areas. We proceeded with three approaches. In one apprach we looked at bill summaries (the \emph{materia} field in bill histories) in order to identify those dealing with agriculture (a key Chilean industry); with international trade (Chile is an export-led economy); and foreign affairs (easily identifiable). Another approach was to look at the a priori categorization that draft laws receive upon introduction (called the \emph{boletín}), which points to a general thematic area. Easily-grasped themes include agriculture, foreign affairs, mining, and taxes (copper remains Chile's main commodity, a State monopoly financing a substantial portion of public spending). The third approach was to rely on a classification of bills into levels of legislative significance, which the literature on legislative politics has long accepted to be useful. Mayhew’s (1991) outstanding criteria, extended by others (e.g., Cameron 2000; Clinton and Lapinski 2006), is difficult to apply in other contexts because that work relied on the existence of sophisticated media analyses not available in the Latin American setting. We take the variable from work we published elsewhere (cites removed for the sake of anonymity), following the criteria in Molinelli et al. (1999), and maintains three levels of significance. This classification is conceptually similar to Mayhew's (1991). The three levels identified are: (1) landmark legislation, (2) important legislation, and (3) minor legislation. The appendix provides further elaboration of what each level includes. While this classification is available for a sample of the years included in our study (years 2006-2008 and 2010-2013) it covers the first three years of two administrations, one from each of the two coalitions dominating Chilean politics.


%The third approach was to rely on a classification of bills into levels of legislative significance, which the literature on legislative politics has long accepted to be useful. Mayhew’s (1991) outstanding criteria, extended by others (e.g., Cameron 2000; Clinton and Lapinski 2006), is difficult to apply in other contexts because that work relied on the existence of sophisticated media analyses not available in the Latin American setting. We take the variable from Palanza (2020), who follows the criteria in Molinelli, Palanza and Sin (1999), later extended in Palanza and Sin (2014), and maintains three levels of significance. This classification is conceptually similar to Mayhew's (1991). The three levels identified are: (1) landmark legislation, (2) important legislation, and (3) minor legislation. The appendix provides further elaboration of what each level includes. While this classification is available for a sample of the years included in our study (years 2006-2008 and 2010-2013) it covers the first three years of two administrations, one from each of the two coalitions dominating Chilean politics.

%\textbf{INCLUDE THE FOLLOWING IN THE APPENDIX:}
%Level 1 – Landmark legislation: This category includes only major pieces of legislation that establish or profoundly reform areas of legislation that have broad effects across the nation. Only extensive and profound reforms to these laws enter this category (narrow reforms are placed in the next category). 
%Level 2 – Important legislation: This category includes legislation with varying degrees of importance, yet that ultimately falls short of producing the broad effects of landmark legislation. It includes legislation that reforms aspects of landmark legislation, and legislation that regulates important issues but is constrained in terms of scope. 
%Level 3 – Minor legislation: This category includes pieces of legislation that are mostly symbolic or that establish issues that are important only for a very small group of individuals, with minor consequences for the broader population. 

All three approaches are somewhat crude and far from complete---a systematic classification of bill issue areas and importance is a difficult task beyond the scope of our current project. Importantly, the re-estimation of the models using these measures and subsets of bills by issue area and importance (which we report in our on-line appendix), confirms that the results in the text are robust. In general, the estimate for friendly committee chairs remains positive across issue areas, even if the small $N$ plays against coefficient significance. And coefficient estimates suffer no substantial change when bills of variable importance are isolated. 

We also elaborate in the closing section Reviewer \#1's point that urgency applicable to any policy area has more strategic value than the U.S.\ fast-track, which is restricted to international trade.

% The development of a classification of bills' precise issue (and the desirable and related measure of interest, bill importance) is beyond the scope of this paper. Since the model , we kept it in the text. The appendix reports the effect on our key regressor, explains how we classified bills from the summary of the draft bills.

% Reviewer: Another aspect to be considered by the author is which bills require more presidential attention at the point when he uses his urgency authority. Since political preferences are the main explanatory factor driving the use of the urgency provision, it is important to consider that laws are not created equal. Among other features, the policy areas are informative about how costly the bills are and how persuasive their costs will be on the presidential decision to use the urgency provision. The author considers these bill-specific features, using the variable "hacienda referral", to control the effect of the policy area. However, it would be interesting to test this effect using a variable that discriminates the policy areas better.  This urgency authority can be used to expedite legislation in all policy areas in Chilean and Latin American presidential systems. Therefore, it increases the strategic value of this provision in these presidents' eyes, as compared to the US case, where it restricts international trade agreements.

\subsection{Consider shortening the discussion of the findings and expanding the discussion of the implications} In light of Reviewer \#1's suggestion, we have shortened our discussion of results and elaborated further the implications in the final section. As mentioned, we included a discussion of the differences between decree and urgency powers. It was particularly interesting to think about the implications of our analysis for the scholarship on comparative institutions, and the tool executives have toadvance their agneda. Thus, we explored the implications of the argument we present in the manuscript in relation to literature we discussed on presidential powers. We thank reviewer \#1 for pointing us in the direction of thinking deeply on how these findings contribute to filling the gap in the literature on urgency powers and on what it contributes to the literature on restrictive procedures and the closed rule. The new version of our discussion also incorporates directions for future research, highlighting where our work leaves off and directions in which future work may take the agenda. 


%Results and discussion. I recommend the review of the last two sections in order to make them more balanced. In section 5.2, the author provides an extensive description of the findings, concentrated mostly on estimated coefficients. In turn, the author spends only 1.5 pages discussing the implications of this study!
 

%To get this manuscript ready for publication, the author should review section 6 in order to explore more carefully the implications of this study in relation to literature previously discussed. A deeper discussion of whether these findings contribute to filling the gap, related to the approach to urgency authority as a procedural mechanism, is expected. Furthermore, the implications of this analysis for presidential and legislative scholarship should be developed properly, particularly to advance some comparative conclusions as proposed in this manuscript.


\section{Responses to Reviewer \#2}

%While the article is presented as a general theory of urgency motions in Latin America, the model details and results explain the Chilean case. Whether urgency procedures in other countries of Latin America serve the same function and conform to their model remains unproven. Consequently, it is not clear to me that the article generalizes the use of urgency in all (or even most) presidential regimes. That said, I do see merit in the way the authors connect the Chilean case to similar rules across Latin America, as well as the contribution to the broader comparative legislatures literature. However, the scope conditions of their results should be more clearly noted.

\subsection{Does the article generalize the use of urgency for countries in Latin America? } Reviewer \#2 rightfully pointed out that the scope conditions of our theory were not clear throughout the manuscript, although he/she saw merit in the way we ``connect the Chilean case to similar rules across Latin America, as well as the contribution to the broader comparative legislatures literature''. In order to tackle this comment, we highlight the following: First, the case of Chile resembles that of the United States, in the sense that in Chile the president assumes the role of the US Rules Committee so that bills (Chile) or treaties (US) cannot be amended on the floor. In this sense, the Chilean case is important because it shows the implications of Moe and Howell's proposal for the US. Although urgency authority in other countries in Latin America does not serve the same exact function as in Chile, our broader objective is to show that urgency is not only about the most obvious effect, which is to speed up the approval of bills. If we consider the case of Chile, where urgency does not speed bills along, we find that its effect is procedural. Thus, the more general point we want to make is that in analyzing fast track authority in Latin America, procedural considerations must be taken into account to understand the breadth of what is at stake. We clarified the scope of our theory along these lines, and we thank Reviewer \#2 for bringing this to our attention.
 
% We describe the similarities and differences in urgency authority in seven Latin American constitution.   The provisions in the case of Chile that invoke a closed rule for urgent bills, actually equating urgency authority and the fast-track authority of U.S. presidents; (3) model fast-track authority in order to identify its determinants; (4) test hypotheses on these determinants with Chilean legistative data. We also expanded section 6 (Discussion) to list verification of sub-constitutional institutions in Brazil, Colombia, Ecuador, Mexico, Paraguay, and Uruguay among the items in the future to-do list---verify if there is any hint of closed rule there too when a bill becomes urgent. (Ges might find some bits of this evidence, we will also elaborate it in section 6).
 

%The presentation of the model is interesting but, in my view, too long. The intuition is straightforward and the formalization is, as they write, inspired in the restrictive procedure model of Dion and Huber. Much of what is in pages 8 to 16 could be presented to readers in half the space. In particular, the teacher/student example, and Figure 1, seem as part of a larger project rather than strictly germane to this paper. The authors could add this as a supplemental file rather than in the paper.

\subsection{Consider shortening the model section, remove the teacher/student example, and Figure 1}
We value this suggestion, as we see how it enhances the contributions of our manuscript and more appropriately draws attention to the key parts of our argument. We have shortened the model section by half, as suggested by Reviewer \#2. We have removed the example, together with Figure 1. We originally thought we needed to include this example to provide an intuition of how the procedure works, but we realized, as Reviewer \#2 pointed out, that what the example did, rather, was help situate the model within a larger project, but was otherwise confusing. We expect the current version to be much more amenable to all readers while maintaining the main technical points. 


%While I like the paper, the hypotheses presented at the end of section 4 seem odd. The first hypotheses, that fast track bills should be more prevalent than standard ones, would only be true if all standard bills amended on the floor would be then shifted away from the position of the president or her party. More important, already the descriptive information in Table 2 shows that only 37\% of all presidential bills are urgencies and about 50\% of all presidential bills approved have urgencies. So, Hypothesis 1 is both commonsensical and likely false. It also makes little sense theoretically, because we do not know ex ante the number of bills that (using figure 4) would fall in each of the different profiles. If the theory is correct, the rate of urgency would also depend on the likelihood of introducing bills in each profile and, to some extent, that would be dependent on the status quo that are available. In fact, the theory could be used in a way that urgency rates are used to predict the likely profile faced by different presidents (that, of course, falls outside of what the authors are required to do in this paper). In any event, hypothesis 1 is simplistic, theoretically incorrect, and empirically false.

%We mention that empirical implications on gatekeeping and on policy outcomes could be derived from the model, but our data cannot test them.


% Hypotheses 2.a and 2.b are also problematic. I fully agree that committee chairs aligned with the president should elicit higher urgency rates. But, shouldn't this be also the case because the bills sent to committees chaired by the opposition will alter the bill away from the president? So, isn't it more likely that bills in opposition committees are allowed to be voted under open-rule so that the majority can restore the original intent of the president and her party? So, any situation in which the chairs are "imperfect" gatekeepers of the preferences of the president and her party will result in the adoption of the standard process. I can flip the argument and say that open rule is a way to bring back "unruly" chairs and to discipline them. The problem with the current Hypothesis 2.a and 2.b is that they focus on the capacity of the president to enforce their preferred outcome but disregards the need of the president to rain down on unruly legislators. As a result, the wording of H2 is terrible: "standard bill consideration does not occur when the chair of the reporting committee belongs to the president's party". This is, again, theoretically unwarranted and empirically false. We can see in the statistical analyses that coalition chairs aligned with the party increase urgency by [.825, .874, .847]. Given that the average rate for approved bills is 50\% (and I am not even considering all bills, only approved one), the rate of urgency is exp(0+.825)/(1+exp(0+.825))= 69\%. This is pretty far from the 100\% required in the wording of H2. The authors indicate that the statistical results "cannot reject H2" but this is clearly wrong, as the proper comparison is not against the baseline (mean of the model) but against this extreme 100\% they describe in Hypothesis 2. Again, as in H1, the wording of H2 is simplistic, theoretically unwarranted, and empirically false.


\subsection{Consider reformulating the Hypotheses taking into account that when a bill goes to a committee far from the president's preferences, the probabilities of fast track should be lower because an open rule means the bill is moved back to the floor median} Reviewer \#2 raised a very important point about our hypotheses which led us to reformulate them. First, Reviewer \#2 noted that, if only half of presidential initiatives that became law received the ``urgency'' denomination, then it could not be true that fast track bills were more prevalent than standard ones. This comment was very helpful in helping us realize that from the model, we could not derive a specific hypothesis about frequency of urgencies, but instead, that our hypotheses were related to the conditions under which we should see bills under the fast track mechanism. 

Second, Reviewer \#2 provided a most valuable suggestion about the reformulation of our hypotheses.  He/she indicated that we should consider that bills that come from ``opposition committees are allowed to be voted under open-rule so that the majority can restore the original intent of the president and her party''. This is an excellent point that we overlooked by focusing exclusively on the ability of the president to enforce bargaining with ``friendly'' committee chairs. The new hypotheses reflect these helpful suggestions and incorporate Reviewer \#2's suggestion that presidents will allow for an open rule when ``the chairs are "imperfect" gatekeepers of the preferences of the president'', and that they will refrain from fast track procedures when they need ``to bring back unruly chairs and to discipline them''. Thus, the revised manuscript offers hypotheses about the effect preference overlap between the president and the committee chair on the probability of fast tracking a bill, and of preference divergence between the president and the committee chair on the probability that bills are considered using an open rule. We test these hypotheses in the empirical section,and found suport for them.

We thank Reviewer \#2 for these helpful comments that allowed us to tighten the manuscript by better connecting theory, implications, and empirical analysis. 

 


%Explanation: We re-worded the hypothesis in probabilistic terms. It now reads thus: "Other things constant, a fast-track is no less probable when the chair of the reporting committee belongs to the president’s party than when they belong to different parties." We did not phrase it as more probable because the condition underlying the hypothesis is necessary but insufficient---something that we have made explicit in the appendix.

%The odd thing about H1 and H2 is that the paper is doing fine without them. Multiple referrals are likely explained by the higher likelihood that the president needs to rain on unruly chairs rather than by the need to protect the original intent of the bill. 

\subsection{Consider the possibility that the president uses multiple referrals to rein in unruly chairs rather than due to the need to protect the original bill}
Point well taken. Again, Reviewer \#2 makes us look at the possibility of unruly chairs and the mechanisms the president uses to rein them in. We agree with Reviewer \#2 that multiple referrals happen because the president wants to rule over undisciplined chairs. We have incorporated these thoughtful considerations in the hipotheses and more generally, in the manuscript. 

%The same with important committees. If there were any data available about committee amendments, it would make more sense to test whether bills "not modified" in committee have much higher rates of urgency. The 100\% urgency rate (still too extreme for my taste) seem to be more likely when bills of the president see no modifications at the committee stage rather than when the committee chair belongs to the president party. In any event, the hypotheses of the paper are in serious need of revisions.
\subsection{Test whether bills "not modified" in committee have much higher rates of urgency.}
Reviewer \#2 pointed out that we should expect higher rates of urgency when bills are not modified in committee (since the bill would supposedly reflect the president's preferences perfectly) than when the chair of the committee belongs to the president's party. This is a good insight that merits further investigation. Presently, we do not have data on committee amendments, and to put such data together is an enormous task that is beyond the scope of what we set out to do in this paper. We inquired about the possibility of collecting this data, but, from our interviews with committee staff, it is not clear that the data could be easily collected. To carry out the analysis of each bill reaching a committee and the amendments it undergoes, is a labor intensive task that requires qualitative assessment and coding of a large number of bills. We leave this investigation for future work, as it is a research project of its own. We thank Reviewer \#2 for calling attention to this kind of data because this data will not only help our arguments regading urgency authority but will also be useful to understand the relations and bargaining between parties, coalitions, and opposition. 
 
%we are not convinced that Reviewer \#2's hypothesis is worth testing. The assumption underlying this idea is that the bill represents presidential preferences exactly,  \textbf{NOTE: please check wording}as we think that it is equally reasonable to assume that the president anticipates that some bargaining will take place, and that she places policy farther so as to provide room for bargaining. While we appreciate this comment, and we may have tested it had the data been more readily available, we refrain from doing so.


%Finally, the fact that the authors observe an electoral cycle in the use of urgencies either indicates that there is a temporal dimension, beyond the procedural management they report, or that the frequency of bills in each profile is changing over time. This may help the authors to inform readers about the actual "urgency" aspect of the bill. While I agree that procedural benefits may be the most important reason to use urgency motions, there seems to be a temporal dimension that deserves to be noted.
\subsection{Is there a temporal dimension in the use of fast track authority?}
Reviewer \#2 asks whether there is an electoral cycle effect in the use of fast track authority. Might electoral pressure trigger urgencies more often? This is an important point we had not considered. In fact, in 2007 legislators proposed to limit the use of urgency authority during the period prior to an election. They suggested that bills marked urgent during the 90 days immediately preceding an election needed to be approved by 2/3 of the members of the chamber to be effective, or that the use of urgency authority be forbidden during the 30 days prior to an election. While these proposals have not prospered, the use of urgency power during electoral campaigns remains to be considered inappropriate (Soto Velazco 2016). We tried to address the point with a dual approach. One portrays the temporal dimension of urgencies explicitly, plotting their frequency throughout consecutive years (figures are reported in our on-line appendix). No clear pattern in urgency usage along the four-year cycle emerges. The other is to re-estimate the models with an electoral cycle control. The resulting coefficient (also reported in our appendix) indicates a slight linear drop, other things constant, in the probability of a fast track as the next election nears. Most important is that all other coefficients experience no substantial change compared to the model controlling for the election year only.

%With this in mind, we opted for the simpler model not controlling the election cycle, adding a cross-reference to the on-line appendix while expanding our discussion of this topic. 
%We deal with an important feature of the legislative process. Our analysis, however, only inspects bill histories in Chile on the surface. Numerous puzzles remain and their study promises an exciting agenda of future research. The referee highlights one dimension deserving attention. We do not elaborate it in the text due to lack of space and in order not to lose focus. Interested readers will hopefully find the discussion in the appendix interesting and work to shed further light on the urgency authority (all our data and code will be posted on-line if the paper gets published).

\section{Reviewer \#3:}

\subsection{Consider making the first half of the paper more ``reader friendly''}
Reviewer \#3 correctly critiques that we assume the reader has a deep understanding of the intricacies of the legislative process and executive-legislative relations in the US and Latin America. We appreciate her/his suggestion to remove all the non-essential discussions of procedures. We revised not only the first half, but all the paper with the objective to tighten it up, and make it more readable. Reviewer \#3 also pointed out that even the example delves into technical issues very quickly. We dropped the example from the revised manuscript (as Reviewer \#2 also suggests) and we also tried to keep technical jargon at bay. Following Reviewer \#3 suggestion to remove part B of Table 2, as well as the related text discussion, we moved that section to the appendix. The manuscript reads much better after this change. We also decided to add information of urgent bills by presidency. Thanks to Reviewer \#3's recommendation we think the current text engages the reader better and makes the contribution of our paper more straightforward.   

	
%The primary concerns I have are three-fold: it is difficult to read much of the first half of the paper given the heavy reliance on technical and procedural explanations, the generalizability that results from the formal model is not made particularly explicit and leaves me wondering about broader scope conditions, and the contention that this is an exploration of the "procedural" effect of using urgency authority actually leaves me wondering about political influences and effects that are not addressed.

%On the first big issue, there is a lot of assumed understanding of technical procedures facing a bill's path through the legislative process and in executive-legislative interactions. I do not think the currently written text is sufficiently general enough for the average (non-Americanist institutions) reader. I would recommend trying to make the text more reader-friendly by stripping down non-essential discussion of the nuts and bolts of legislation construction and procedures and instead using more plain English. In a similar way, the "example" begins as if designed to engage a reader on a non-technical level but quickly devolves into a relatively high degree of technicality that segues into the game. Not everyone reading will be well versed in game theory and/or legislative procedure, and paying attention to how to engage everyone else would be tremendously useful in making clear the bigger contribution of the piece.

%Finally, the information in Part B of Table 2 and the related in-text discussion strikes me as a footnote plus appendix type of discussion. It is distracting from the main point and only one category is considered in the analysis, so I don't think including it in the body of the text makes much sense.





% On the second big issue, I am not a formal theory person, so I have limited ability to comment on whether the model sufficiently and appropriately captures the game and possible outcomes. However, I do question whether the model is generalizable to any cases in which there are restrictions (even modest ones) that limit the number or scope of bills a president may be able to fast-track. Does this really only explain the US, Chile, and maybe another case or two? How much modification would be required to accommodate restrictions on a president's ability to fast-track legislation via urgency authority in the plenary arrest, automatic adoption cases, or Mexico? Is that game capable of producing equally compelling hypotheses that might be tested via comparative statics analysis? Presumably hypothesis 1 cannot possible be theoretically defensible in a system with restrictions on this form of presidential authority.

\subsection{Is the model generalizable to cases that restrict the number or scope of bills a president is able to fast-track?}
This is an important point, and one that  Reviewer \#1 also emphasized. We address this issue in 1.1.


%On the third big issue, I am not certain if this is a byproduct of the model or the theoretical lens of the author(s), but it strikes me that it is not only important to know how often presidents invoke this authority but also whether doing so makes them more likely to get preferred policies passed (unamended). Presidents presumably do not utilize urgency authority unless they expect it to work out in their favor, either to pass the proposed legislation or obtain bargaining power in another piece of proposed legislation. Why are the hypotheses only about the likelihood of standard versus urgent bill consideration without any expectations regarding the likelihood of passing urgently-considered bills?

\subsection{Do presidents get more of their preferred outcomes when using urgency authority?}
Reviewer \#3 asks fundamental questions here. One, in the summary paragraph, is general: what are the political effects of the procedure that we analyze? Another, is particular: why does the manuscript not address hypotheses on the likelihood of passing urgent bills unamended and test them? In response to the first question, we have added reference to, and discuss the approach by McCubbins Noll and Weingast/Thies/Cox-McCubbins, among others, that suggests that institutional procedures are instruments of political manipulation, i.e., their mere existence affects behavior through anticipation. It is worth noting that ths version of the manuscript (per suggestion of Reviewer \#2) does not look at how often urgency power. Instead, we focus on the conditions that trigger its use. Our hypotheses now state more clearly this concern: when presidents are close to the committee chair and can expect their policy preferences to be upheld, they will fast track the bill, and they will choose not to do so when they are not close to the committee chair and expect the committee outcome to counter their preferences (and therefore fast tracking what comes out of that committee would only disable any opportunity to alter the content of the bill as reported by the committee). This is not to say that analyzing the content of bills and the amendments they undergo would not be useful. But, as we explain in 2.5 in response to a comment by Reviewer \#2, to carry out the analysis of each bill reaching a committee and the amendments it undergoes, is a labor intensive task that requires qualitative assessment and coding of a large number of bills --a task that is beyond the scope of what we set out to do in this paper. We leave the investigation of the "form" of policy effects for future work, as it is a research project of its own. This paper focuses on providing evidence that urgency authority is, in fact, a procedural maneuver akin to the closed rule. We added a paragraph at the start of section 1 to make this explicit (there is no contention that this is an exploration of the policy effects of using the procedure). We also expanded section 6 (Discussion) in order to spell out an untested assumption in our argument: that the closed rule in fact shields reports from further amendment, and what testing the missing pieces would involve.


%Furthermore, I understand that a formal model requires oversimplification of the real political world and that the theoretical framework is meant to highlight procedural effects of the use of urgency authority, but I am troubled by the lack of attention to many political consideration that likely matter in understanding presidents' use of this type of authority. First, is the decision to use urgency authority totally costless to the president from a political/public opinion standpoint? The author(s) includes approval ratings from before bill initiation, but it seems like this could be more of a direct theoretical discussion. For example, it is true that presidents with better approval may need to rely less on this authority to push their agenda, but Latin American presidents are often faulted for doing the opposite (in contrast to American presidents). I can understand that maybe it is costless in some places where voters likely do not have a great understanding of procedural intricacies that presidents use to enact their agenda. But in a case like Chile, where legislators are fairly highly professionalized, one might assume that opposition legislators would call foul to try and damage presidential popularity, particularly on specific issues or in the run-up to new elections. How does this consideration for a president affect her/his decisions?

\subsection{What is the effect of public opinion on the president's decision to use fast track authority?}
Reviewer \#3's question is about how public opinion can affect presidential calculations when deciding to declare bills urgent. She/he suggests that whereas in other countries in Latin America voters are unlikely to understand procedural intricacies, "in a case like Chile, where legislators are fairly highly professionalized, one might assume that opposition legislators would call foul to try and damage presidential popularity, particularly on specific issues or in the run-up to new elections." Reviewer \#3 requests a theoretical discussion to complement the analysis we present of approval ratings. In the revised manuscript, where we go over how presidential approval might reduce reliance on urgency authority (presidents more easily get what they want in the assembly) or increase it (because popular presidents might get better reports from the average committee chair), we included the variable Presidential Approval (p.18) precisely to test whether support from public opinion has an effect on the decision to fast track bills. 

%The question remains, however, whether Chilean legislators would play the public card to dissuade the president from using urgency authority, and given the broad, largely accepted powers of the Chilean presidency, and the fact that the prerogative is constitutional and perfectly legal may be the reason why urgency power does not seem to make the headlines that often or cause much upheaval. 

%\textbf{Nota: Creo que lo que está abajo no corresponde. En cambio, dejo redactado que hicimos controles con presidential popularity} 
%WHY WOULD THIS BE IMPORTANT FOR THIS QUESTION? We also re-estimated regression models with subsets of bills by broad issue areas. Classification of bills' precise issue areas (and a related measure of interest, bill importance) is a difficult task that is beyond the scope of this paper. In order to verify robustness of the model that just controls for Finance-comittee-referral in the text, ee relied on a somewhat crude and incomplete issue categorization. The appendix explains how we selected subsets of bills, inferring the issue area through the committee they were referred to as well as an incomplete categorization relying on the thematic summary of the draft bill. The appendix reports results. The small N plays against coefficient significance, but in general the estimate for co-partisan committee chairs remains positive across issue areas, raising confidence that the model in the text is robust to such controls.


%Second, while there is a consideration of the congressional cycle, there is no mention of election-related cycles. However, it seems like from a procedural standpoint new presidents should be less likely to exercise this authority (e.g. they have limited familiarity with procedures and powers) than those with some experience. The absence of the potential for immediate re-election in Chile means that during the time period under study, all presidents are essentially "brand new" to the democratic executive powers at their disposal. I suspect there might be a curvilinear relationship between time in office and propensity to use this authority widely, with inexperience depressing its use early on and upcoming elections depressing its use later on. How does time matter regarding expectations of how presidents use this power? Including time from/to executive and/or legislative election might help here.

\subsection{What is the effect that the electoral cycle has on the president's decision to fast track bills?}
We thank Reviewer \#3 for bringing up this issue and providing many excellent suggestions. We added a section in the appendix discussing temporal effects in urgency usage along the election cycle. Note that Reviewer \#2 also brings up the effect of time and of the electoral cycle (as we discuss in 2.6). Here we reiterate part of that discussion and point out that in 2007 legislators proposed to limit the use of urgency authority during the period prior to an election, making clear that it is an issue of concern. Ultimately, however, that reform did not prosper (Soto Velazco 2016). We tried to address the point by re-estimating the models including a control for the electoral cycle. The resulting coefficient (reported in our appendix) indicates a slight linear drop, other things constant, in the probability of a fast track as the next election nears. Most important is that all other coefficients experience no substantial change compared to the model controlling for the election year only.

%We also estimated and reported in the appendix the model with this control (and its square, to capture a non-linear pattern). The effect is statistically significant but small, indicating a drop in urgency use as the election nears. Other coefficients remain virtually unchanged with the addition of this control,. 
%Third, doesn't presidential experience also matter? Presidents who were previously part of a cabinet may have a very different take on the relationship between the executive and legislature than former legislators. Also, in the reverse direction, Chilean presidents sometimes move into legislative offices after they leave the executive branch: does Frei's experience wielding this authority affect his response to it when he becomes a senator later on down the line? Considering a set of controls for previous experience of the president (legislative branch, executive branch, or none) and previous experience of sitting legislators (or perhaps just the committee or coalition chairs) should help to address this.
\subsection{What is the effect of presidents and legislators' previous experience on the use of fast track authority?}
This is an interesting point that had not occured to us before Reviewer \#3 brought it up. As we know, Frei and Lagos had been legislators prior to assuming presidential office, and Bachelet and Piñera had not. After the end of each of their presidential terms, only Frei assumed legislative office during the years under analysis. We tried to  capture prior presidential experience among these four presidents by using fixed (Model 3) and mixed (Model 4) effects. We have also included a temporal breakdown of urgencies by presidency as part of Table 1 and  discuss this propensity in the text. Additionally, Reviewer \#3's question also made us consider what is the effect of legislators' previous experience. Do legislators' different trajectories prior to their legislative terms affect presidential use of the urgency prerogative? We do not have the data to analyze this question right now, although we are looking forward to extending this analysis in the future. It is especially wonderful to receive these comments and engage in them in extensions and future papers, as we believe this question deserves an answer, one that can lead us to better understand the effect of learning on the actual use of institutions.  This is anWhile it would be interesting to know, and our guess would be that it does not, we do not have the data to test such claims at this point, and their collection seems to exceed the goal of this paper.
 
 %Additionally, in the Appendix, Figure A.1 show that presidents without legislative experience relied way more on urgencies than those with experience. But the available data cannot answer if this is the explanatory factor behind the surge, or some other factor. 
%\section{Cover letter for RnR}
%\label{sec:orgf3fbbd7}
%October 10, 2018

%Lanny W. Martin
%Professor of Political Science
%Comparative Politics Editor, Journal of Politics

%Department of Social and Political Sciences
%Bocconi University
%anny.martin@unibocconi.it

%Dear Dr. Buhaug: My co-authors and I have revised our manuscript and are re-submitting it for your consideration and for a second review. 

%We addressed all concerns by reviewers, either in the text or in a new on-line appendix. This letter explains how we have done it. 

\end{document}
