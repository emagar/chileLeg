% Created 2019-03-04 Mon 10:17
% Intended LaTeX compiler: pdflatex
\documentclass[article,letterpaper,times,12pt,listings-bw,microtype]{article}
\usepackage[utf8]{inputenc}
\usepackage[T1]{fontenc}
\usepackage{graphicx}
\usepackage{grffile}
\usepackage{longtable}
\usepackage{wrapfig}
\usepackage{rotating}
\usepackage[normalem]{ulem}
\usepackage{amsmath}
\usepackage{textcomp}
\usepackage{amssymb}
\usepackage{capt-of}
\usepackage{hyperref}
\usepackage[margin=0.5in]{geometry}
\usepackage{xr}
\externaldocument[app-]{urge14appendix}
\author{Double blind review}
\date{\today}
\title{Resubmission of "Presidents on the Fast Track: Fighting Floor Amendments with Restrictive Rules" to the \emph{Journal of Politics} (Ms. No. 182014)}
\hypersetup{
 pdfauthor={Double blind review},
 pdftitle={Resubmission of "Presidents on the Fast Track: Fighting Floor Amendments with Restrictive Rules" to the \emph{Journal of Politics} (Ms. No. 182014)},
 pdfkeywords={},
 pdfsubject={},
 pdfcreator={Emacs 24.5.1 (Org mode 9.1.7)}, 
 pdflang={English}}
\begin{document}

\maketitle
\tableofcontents

\newpage

\noindent
We cut reviews' issues into bite-size numbered paragraphs. Our responses in bullets follow each paragraph.

\section{Reviewer \#1:}
\label{sec:org3522269}
\begin{enumerate}
\item This analysis seeks to widen the comparative analysis of executive influence in lawmaking by highlighting the procedural consequences of urgency authority. The author claims that fast-track and urgency prerogatives granted to US and Latin American presidents, respectively, are comparable, since the rationale driving the use of both provisions is to expedite the legislative process.
\label{sec:org7ea5322}
\begin{itemize}
\item \textbf{Actions taken}: None, reviewer asks nothing from us here.
\end{itemize}
\item After contrasting the urgency and fast-track prerogatives, in a subset of Latin American countries and the US, the author presents a formal model of the fast-track authority as a game of restrictive procedures and the resulting equilibrium. The main question in this examination is, "Under what conditions will a bill be fast-tracked?" The main variable of interest is the preference proximity between the committee chairs and the president. The hypotheses stemming from the formal model are tested using data from Chile (1998-2014).
\label{sec:org3dacab3}
\begin{itemize}
\item \textbf{Actions taken}: None, reviewer asks nothing from us here.
\end{itemize}
\item This manuscript contributes effectively to the scholarship on inter-branch relations, in lawmaking in separation of power systems. Its main contribution is the development of a formal model to generate hypotheses for the comparative analysis of these relations, considering different mechanisms granted to presidents to control legislative decisions.
\label{sec:orgb1504c6}
\begin{itemize}
\item \textbf{Actions taken}: None, reviewer asks nothing from us here.
\end{itemize}
\item It is an interesting and well‐executed manuscript, that is of interest to the JP readership. Thus, I recommend its publication after minor revision. Below, I detail my comments and suggestions to the author.
\label{sec:org4e12a76}
\begin{itemize}
\item \textbf{Actions taken}: None, reviewer asks nothing from us here.
\end{itemize}
\item 1.    Organization and framing. The manuscript is well organized and well written, despite a few typing errors. The numbering and organization of sections are appropriate, contributing to the clarity and consistency of the manuscript's argumentative structure.
\label{sec:org756ff24}
\begin{itemize}
\item \textbf{Actions taken}: Conceded, we did our best to fix typing errors in the text.
\item \textbf{Explanation}: If the manuscript were accepted for publication, we will thoroughly copy-edit it with the aid of native English speaker to improve the writing.
\end{itemize}
\item 2. Arguments and model issues. The claim that the fast-track and urgency authorities are equivalent is convincing. The author's formal model makes clear both the similar rationale behind the two mechanisms, and how they affect the president's and legislators' moves in the legislative arena.
\label{sec:org8b65dd3}
\begin{itemize}
\item \textbf{Actions taken}: None, reviewer asks nothing from us in this start of paragraph.
\end{itemize}
\item However, it is important to consider, at least theoretically, some conditions that might modify the president's decision to resort to urgency authority. Since the focus of the manuscript is on the procedural effects of the urgency authority, mostly in speeding up the legislative process, it matters whether it is the only provision that the president counts on to generate decisions faster, or if other tools are available. This is important, given that the question driving this analysis is, "Under what conditions will the president resort to his urgency authority?"
\label{sec:org389a939}
\begin{itemize}
\item \textbf{Actions taken}: Elaborated in the discussion.
\item \textbf{Explanation}: The introduction puts urgency authority in the same level as executive decrees---another presidential tool for unorthodox lawmaking. The closing section (Discussion) retakes the decrees-urgencies tandem---two features of Latin American presidentialism. In particular, we question if in cases sharing both institutions (Brazil, Colombia, Ecuador), decrees might in fact reduce urgency attractiveness. Figeiredo and Limongi report that Brazilian presidents rarely declare bills urgent, then quote them saying that there is a natural preference for more versatile decrees. We also mention that the contrary may hold if policy by statute were more valuable to presidents than by mere excutive decree.
\end{itemize}
\item Different from the US president, some Latin American presidents count on constitutional decree powers, which allow them to issue decrees having the force of law without prior authorization of the Congress. Usually limited to specific policy areas, this decree power allows the president to pursue his preferences and overturn legislative resistance to them.   As is known, this kind of provision grants the president first-move advantages to change the "status quo" and implement a decision close to his own preferences. This forces legislators to respond to the president's decision instead of the previous status quo. In addition, as emergency decrees, they contain provisions that make introducing floor amendments more restrictive and, consequently, expedite the legislative process. Therefore, are the odds that the president will qualify a bill as urgent the same when he has decree power? I suggest the author discuss this point, even if only to make clear that consideration of this condition will not impact either the explanatory power of the model or the results of the analysis. Empirical analysis is done only for the Chilean case where, as in the US, constitutional decree power is not granted. However, it is important to consider this point, since this manuscript seeks to offer a model for comparative analysis.
\label{sec:org9bfffd1}
\begin{itemize}
\item \textbf{Actions taken}: Elaborated in the discussion.
\item \textbf{Explanation}: Done in closing section (Discussion), expanded along the lines that the referee suggests.
\end{itemize}
\item Another aspect to be considered by the author is which bills require more presidential attention at the point when he uses his urgency authority. Since political preferences are the main explanatory factor driving the use of the urgency provision, it is important to consider that laws are not created equal. Among other features, the policy areas are informative about how costly the bills are and how persuasive their costs will be on the presidential decision to use the urgency provision. The author considers these bill-specific features, using the variable "hacienda referral", to control the effect of the policy area. However, it would be interesting to test this effect using a variable that discriminates the policy areas better.  This urgency authority can be used to expedite legislation in all policy areas in Chilean and Latin American presidential systems. Therefore, it increases the strategic value of this provision in these presidents' eyes, as compared to the US case, where it restricts international trade agreements.
\label{sec:org315e69e}
\begin{itemize}
\item \textbf{Actions taken}: Defended our model and elaborated in the appendix and a footnote.
\item \textbf{Explanation}: We did a classification of bills in broad issue areas and re-estimated the regression on those subsets. In general the estimate for friendly committee chairs remains positive across issue areas, even if the small N plays against coefficient significance. Frankly, the issue classification we relied upon is crude and incomplete. Classification of bills' precise issue (and the desirable and related measure of interest, bill importance) is tricky enough to be beyond the scope of this paper. Since the model controlling for Finance-comittee-referral only seems robust, we kept it in the text. The appendix reports the effect on our key regressor, explains how we classified bills from the summary of the draft bills.
\item \textbf{Explanation} (cont.): We also elaborate the point that urgency applicable to all policy areas has more strategic value than the U.S. fast-track, restricted to international trade.
\end{itemize}
\item 3. Results and discussion. I recommend the review of the last two sections in order to make them more balanced. In section 5.2, the author provides an extensive description of the findings, concentrated mostly on estimated coefficients. In turn, the author spends only 1.5 pages discussing the implications of this study!
\label{sec:orgb13192f}
\begin{itemize}
\item \textbf{Actions taken}: Conceded.
\item \textbf{Explanation}: The revised manuscript elaborates implications in the final section. Section 4.2 (Model specification and results) takes four-and-a-half pages, about three of which are text; section 5 (Discussion) takes almos three pages. We have included the referee's suggestion to contrast decree/urgency in the discussion, and elaborated suggestions by other referees.
\end{itemize}
\item \textbf{check} To get this manuscript ready for publication, the author should review section 6 in order to explore more carefully the implications of this study in relation to literature previously discussed. A deeper discussion of whether these findings contribute to filling the gap, related to the approach to urgency authority as a procedural mechanism, is expected. Furthermore, the implications of this analysis for presidential and legislative scholarship should be developed properly, particularly to advance some comparative conclusions as proposed in this manuscript.
\label{sec:orga4aa713}
\begin{itemize}
\item \textbf{Actions taken}: Conceded.
\item \textbf{Explanation}: The revised closing section (Discussion) explores implications\ldots{} EXPAND AFTER FINAL REVIEW OF DISCUSSION SECTION.
\end{itemize}
\item In summary, I recommend the publication of this manuscript after the review of these minor corrections.
\label{sec:orgfdc2c4f}
\begin{itemize}
\item \textbf{Actions taken}: conceded, the Authors hope that the revisions above have taken care of the Reviewer's concerns.
\end{itemize}
\end{enumerate}
\section{Reviewer \#2:}
\label{sec:orga4b521d}
\begin{enumerate}
\item This is an interesting manuscript on a relevant topic. Urgency procedures are common and understudied in presidential regimes. This article provides evidence that urgency procedures provide procedural advantages to presidents, which not merely speed up approval but, more important, allow presidents to approve legislation with fewer modifications. Consequently, if committee chairs align with the party of the president, urgency rules that force an up-or-down vote on the floor should occur with more frequency. The authors test their argument with data from Chile, with positive findings for their argument. In all, this is a paper that, after some changes, could be of interest to readers of JOP. In all, this is a very promising paper, with sound empirics, that needs some important adjustments prior to publication.
\label{sec:org86801b7}
\begin{itemize}
\item \textbf{Actions taken}: None, reviewer asks nothing from us here.
\end{itemize}
\item While the article is presented as a general theory of urgency motions in Latin America, the model details and results explain the Chilean case. Whether urgency procedures in other countries of Latin America serve the same function and conform to their model remains unproven. Consequently, it is not clear to me that the article generalizes the use of urgency in all (or even most) presidential regimes. That said, I do see merit in the way the authors connect the Chilean case to similar rules across Latin America, as well as the contribution to the broader comparative legislatures literature. However, the scope conditions of their results should be more clearly noted.
\label{sec:org3c830b1}
\begin{itemize}
\item \textbf{Actions taken}: Conceded and expanded the discussion.
\item \textbf{Explanation}: We re-wrote the introduction in order to clarify what we actually achieve in the manuscript: (1) describe similarities and differences in urgency authority in seven Latin American constitutions; (2) uncover sub-constitutional provisions in the case of Chile that invoke a closed rule for urgent bills, actually equating urgency authority and the fast-track authority of U.S. presidents; (3) model fast-track authority in order to identify its determinants; (4) test hypotheses on these determinants with Chilean legistative data. We also expanded section 6 (Discussion) to list verification of sub-constitutional institutions in Brazil, Colombia, Ecuador, Mexico, Paraguay, and Uruguay among the items in the future to-do list---verify if there is any hint of closed rule there too when a bill becomes urgent. (Ges might find some bits of this evidence, we will also elaborate it in section 6).
\end{itemize}
\item The presentation of the model is interesting but, in my view, too long. The intuition is straightforward and the formalization is, as they write, inspired in the restrictive procedure model of Dion and Huber. Much of what is in pages 8 to 16 could be presented to readers in half the space. In particular, the teacher/student example, and Figure 1, seem as part of a larger project rather than strictly germane to this paper. The authors could add this as a supplemental file rather than in the paper.
\label{sec:org5045633}
\begin{itemize}
\item \textbf{Actions taken}: Conceded, we cut the model's exposition to four pages.
\item \textbf{Explanation}: We dropped the example, and compacted the explanation to about half the original length.
\end{itemize}
\item While I like the paper, the hypotheses presented at the end of section 4 seem odd. The first hypotheses, that fast track bills should be more prevalent than standard ones, would only be true if all standard bills amended on the floor would be then shifted away from the position of the president or her party. More important, already the descriptive information in Table 2 shows that only 37\% of all presidential bills are urgencies and about 50\% of all presidential bills approved have urgencies. So, Hypothesis 1 is both commonsensical and likely false. It also makes little sense theoretically, because we do not know ex ante the number of bills that (using figure 4) would fall in each of the different profiles. If the theory is correct, the rate of urgency would also depend on the likelihood of introducing bills in each profile and, to some extent, that would be dependent on the status quo that are available. In fact, the theory could be used in a way that urgency rates are used to predict the likely profile faced by different presidents (that, of course, falls outside of what the authors are required to do in this paper). In any event, hypothesis 1 is simplistic, theoretically incorrect, and empirically false.
\label{sec:orgde19343}
\begin{itemize}
\item \textbf{Actions taken}: Conceded. We have dropped Hypothesis 1 in the shortened presentation.
\item \textbf{Explanation}: The revised manuscript offers a single hypothesis about the effect of the preference overlap between the president and the committee chair on the probability of declaring a bill urgent. It is this hypothesis that the empirical section tests. We mention that empirical implications on gatekeeping and on policy outcomes could be derived from the model, but our data cannot test them.
\end{itemize}
\item Hypotheses 2.a and 2.b are also problematic. I fully agree that committee chairs aligned with the president should elicit higher urgency rates. But, shouldn't this be also the case because the bills sent to committees chaired by the opposition will alter the bill away from the president? So, isn't it more likely that bills in opposition committees are allowed to be voted under open-rule so that the majority can restore the original intent of the president and her party? So, any situation in which the chairs are "imperfect" gatekeepers of the preferences of the president and her party will result in the adoption of the standard process. I can flip the argument and say that open rule is a way to bring back "unruly" chairs and to discipline them. The problem with the current Hypothesis 2.a and 2.b is that they focus on the capacity of the president to enforce their preferred outcome but disregards the need of the president to rain down on unruly legislators. As a result, the wording of H2 is terrible: "standard bill consideration does not occur when the chair of the reporting committee belongs to the president's party". This is, again, theoretically unwarranted and empirically false. We can see in the statistical analyses that coalition chairs aligned with the party increase urgency by [.825, .874, .847]. Given that the average rate for approved bills is 50\% (and I am not even considering all bills, only approved one), the rate of urgency is exp(0+.825)/(1+exp(0+.825))= 69\%. This is pretty far from the 100\% required in the wording of H2. The authors indicate that the statistical results "cannot reject H2" but this is clearly wrong, as the proper comparison is not against the baseline (mean of the model) but against this extreme 100\% they describe in Hypothesis 2. Again, as in H1, the wording of H2 is simplistic, theoretically unwarranted, and empirically false.
\label{sec:orgb6d9906}
\begin{itemize}
\item \textbf{Actions taken}: Conceded, hypotheses have changed.
\item \textbf{Explanation}: We re-worded the hypothesis in probabilistic terms. It now reads thus: "Other things constant, a fast-track is no less probable when the chair of the reporting committee belongs to the president’s party than when they belong to different parties." We did not phrase it as more probable because the condition underlying the hypothesis is necessary but insufficient---something that we have made explicit in the appendix.
\end{itemize}
\item The odd thing about H1 and H2 is that the paper is doing fine without them. Multiple referrals are likely explained by the higher likelihood that the president needs to rain on unruly chairs rather than by the need to protect the original intent of the bill. The same with important committees. If there were any data available about committee amendments, it would make more sense to test whether bills "not modified" in committee have much higher rates of urgency. The 100\% urgency rate (still too extreme for my taste) seem to be more likely when bills of the president see no modifications at the committee stage rather than when the committee chair belongs to the president party. In any event, the hypotheses of the paper are in serious need of revisions.
\label{sec:org1fd7ae5}
\begin{itemize}
\item \textbf{Actions taken}: Conceded and elaborated.
\item \textbf{Explanation}: The revised manuscript has a unique hypothesis. We expanded the final section (Discussion) to discuss the unruly committee/floor implications of the urgency that the referee highlights in this point and the last.
\end{itemize}
\item Finally, the fact that the authors observe an electoral cycle in the use of urgencies either indicates that there is a temporal dimension, beyond the procedural management they report, or that the frequency of bills in each profile is changing over time. This may help the authors to inform readers about the actual "urgency" aspect of the bill. While I agree that procedural benefits may be the most important reason to use urgency motions, there seems to be a temporal dimension that deserves to be noted.
\label{sec:org11080e3}
\begin{itemize}
\item \textbf{Actions taken}: Elaborated in the appendix.
\item \textbf{Explanation}: The appendix portrays empirically the temporal dimension in the use of urgencies, adding a cross-reference in the text (when introducing the regression's temporal controls). We deal with an important feature of the legislative process. Our analysis, however, only inspects bill histories in Chile on the surface. Numerous puzzles remain and their study promises an exciting agenda of future research. The referee highlights one dimension deserving attention. We do not elaborate it in the text due to lack of space and in order not to lose focus. Interested readers will hopefully find the discussion in the appendix interesting and work to shed further light on the urgency authority (all our data and code will be posted on-line if the paper gets published).
\end{itemize}
\item As said, this is a very promising paper, with sound empirics, that needs some important adjustments prior to publication.
\label{sec:org3839cf6}
\begin{itemize}
\item Action taken: Conceded, the Authors hope that they have adequately addressed the referee's objections above.
\end{itemize}
\end{enumerate}
\section{Reviewer \#3:}
\label{sec:orgdbcc5ca}
\begin{enumerate}
\item This is an interesting piece that addresses a gap between the literature on executive power in the US and comparatively: what affects the ability of presidents to wield urgency authority and influence policymaking? Bridging this gap is important and the piece addresses this question by looking at the case of Chile, typically presumed to be a presidential system that is relatively weak in the Latin American context.
\label{sec:org7ef3b06}
\begin{itemize}
\item \textbf{Actions taken}: None, reviewer asks nothing from us here.
\end{itemize}
\item The primary concerns I have are three-fold: it is difficult to read much of the first half of the paper given the heavy reliance on technical and procedural explanations, the generalizability that results from the formal model is not made particularly explicit and leaves me wondering about broader scope conditions, and the contention that this is an exploration of the "procedural" effect of using urgency authority actually leaves me wondering about political influences and effects that are not addressed.
\label{sec:org4cfb701}
\begin{itemize}
\item \textbf{Actions taken}: None, reviewer asks nothing from us here.
\item \textbf{Explanation}: The reviewer lists three issues of concern in this "summary paragraph", developing each in the paragraphs below. We respond to each one below.
\end{itemize}
\item On the first big issue, there is a lot of assumed understanding of technical procedures facing a bill's path through the legislative process and in executive-legislative interactions. I do not think the currently written text is sufficiently general enough for the average (non-Americanist institutions) reader. I would recommend trying to make the text more reader-friendly by stripping down non-essential discussion of the nuts and bolts of legislation construction and procedures and instead using more plain English. In a similar way, the "example" begins as if designed to engage a reader on a non-technical level but quickly devolves into a relatively high degree of technicality that segues into the game. Not everyone reading will be well versed in game theory and/or legislative procedure, and paying attention to how to engage everyone else would be tremendously useful in making clear the bigger contribution of the piece.
\label{sec:orge93bcf9}
\begin{itemize}
\item \textbf{Actions taken}: Conceded, we have simplified the text.
\item \textbf{Explanation}: We dropped the example from the revised manuscript. We also did our best to keep technical jargon at bay.
\end{itemize}
\item On the second big issue, I am not a formal theory person, so I have limited ability to comment on whether the model sufficiently and appropriately captures the game and possible outcomes. However, I do question whether the model is generalizable to any cases in which there are restrictions (even modest ones) that limit the number or scope of bills a president may be able to fast-track. Does this really only explain the US, Chile, and maybe another case or two? How much modification would be required to accommodate restrictions on a president's ability to fast-track legislation via urgency authority in the plenary arrest, automatic adoption cases, or Mexico? Is that game capable of producing equally compelling hypotheses that might be tested via comparative statics analysis? Presumably hypothesis 1 cannot possible be theoretically defensible in a system with restrictions on this form of presidential authority.
\label{sec:org1de3684}
\begin{itemize}
\item \textbf{Actions taken}: Conceded and expanded the discussion.
\item \textbf{Explanation}: We re-wrote the introduction in order to clarify what we actually achieve in the manuscript: (1) describe similarities and differences in urgency authority in seven Latin American constitutions; (2) uncover sub-constitutional provisions in the case of Chile that invoke a closed rule for urgent bills, actually equating urgency authority and the fast-track authority of U.S. presidents; (3) model fast-track authority in order to identify its determinants; (4) test hypotheses on these determinants with Chilean legistative data. We also expanded section 6 (Discussion) to list verification of sub-constitutional institutions in Brazil, Colombia, Ecuador, Mexico, Paraguay, and Uruguay among the items in the future to-do list---verify if there is any hint of closed rule there too when a bill becomes urgent. (Ges might find some bits of this evidence, we will also elaborate it in section 6).
\end{itemize}
\item On the third big issue, I am not certain if this is a byproduct of the model or the theoretical lens of the author(s), but it strikes me that it is not only important to know how often presidents invoke this authority but also whether doing so makes them more likely to get preferred policies passed (unamended). Presidents presumably do not utilize urgency authority unless they expect it to work out in their favor, either to pass the proposed legislation or obtain bargaining power in another piece of proposed legislation. Why are the hypotheses only about the likelihood of standard versus urgent bill consideration without any expectations regarding the likelihood of passing urgently-considered bills?
\label{sec:orgc172133}
\begin{itemize}
\item \textbf{Actions taken}: Elaborated missing pieces.
\item \textbf{Explanation}: The reviewer asks fundamental questions here. One (in the summary paragraph) is general : what are the political effects of the procedure that we analyze? Another is particular: why does the manuscript not address hypotheses on the likelihood of passing urgent bills unamended and test them? We have added reference to and discussion of McNollgast/Thies/Cox-McCubbins's approach that procedures are instruments of political manipulation, then left investigation of the form of those policy effects for future work. The reason is that systematizing the evidence on bill amendment adoption/rejection is far from straightforward---it is a research project of its own. We therefore not investigate that part of the mechanism and concentrate on demonstrating that urgency authority is, in fact, a procedural maneuver akin to the closed rule. We added a paragraph at the start of section 1 to make this explicit (there is no contention that this is an exploration of the policy effects of using the procedure). We also expanded section 6 (Discussion) in order to spell out an untested assumption in our argument: that the closed rule in fact shields reports from further amendment, and what testing the missing pieces would involve.
\end{itemize}
\item Furthermore, I understand that a formal model requires oversimplification of the real political world and that the theoretical framework is meant to highlight procedural effects of the use of urgency authority, but I am troubled by the lack of attention to many political consideration that likely matter in understanding presidents' use of this type of authority. First, is the decision to use urgency authority totally costless to the president from a political/public opinion standpoint? The author(s) includes approval ratings from before bill initiation, but it seems like this could be more of a direct theoretical discussion. For example, it is true that presidents with better approval may need to rely less on this authority to push their agenda, but Latin American presidents are often faulted for doing the opposite (in contrast to American presidents). I can understand that maybe it is costless in some places where voters likely do not have a great understanding of procedural intricacies that presidents use to enact their agenda. But in a case like Chile, where legislators are fairly highly professionalized, one might assume that opposition legislators would call foul to try and damage presidential popularity, particularly on specific issues or in the run-up to new elections. How does this consideration for a president affect her/his decisions?
\label{sec:org10fa7d1}
\begin{itemize}
\item \textbf{Actions taken}: Elaborated in the text and the appendix.
\item \textbf{Explanation}: The revised manuscript mentions how presidential approval might reduce reliance on urgency authority (presidents more easily get what they want in the assembly) or increase it (because popular presidents might get better reports from the average committee chair). We also re-estimated regression models with subsets of bills by broad issue areas. Classification of bills' precise issue areas (and a related measure of interest, bill importance) is a difficult task that is beyond the scope of this paper. In order to verify robustness of the model that just controls for Finance-comittee-referral in the text, ee relied on a somewhat crude and incomplete issue categorization. The appendix explains how we selected subsets of bills, inferring the issue area through the committee they were referred to as well as an incomplete categorization relying on the thematic summary of the draft bill. The appendix reports results. The small N plays against coefficient significance, but in general the estimate for co-partisan committee chairs remains positive across issue areas, raising confidence that the model in the text is robust to such controls.
\end{itemize}
\item Second, while there is a consideration of the congressional cycle, there is no mention of election-related cycles. However, it seems like from a procedural standpoint new presidents should be less likely to exercise this authority (e.g. they have limited familiarity with procedures and powers) than those with some experience. The absence of the potential for immediate re-election in Chile means that during the time period under study, all presidents are essentially "brand new" to the democratic executive powers at their disposal. I suspect there might be a curvilinear relationship between time in office and propensity to use this authority widely, with inexperience depressing its use early on and upcoming elections depressing its use later on. How does time matter regarding expectations of how presidents use this power? Including time from/to executive and/or legislative election might help here.
\label{sec:orgebdf0fa}
\begin{itemize}
\item \textbf{Actions taken}: Defended and clarified.
\item \textbf{Explanation}: We added a section in the appendix discussing the lack of an apparent temporal effect in urgency usage along the election cycle. We also estimate and report in the appendix the model with this control (and its square, to capture a non-linear pattern). The effect is statistically significnt but small, indicating a drop in supreme urgency use as the election nears. Since other coefficients remain virtually unchanged with the addition of this control, we prefer to report the models excluding this control.
\end{itemize}
\item Third, doesn't presidential experience also matter? Presidents who were previously part of a cabinet may have a very different take on the relationship between the executive and legislature than former legislators. Also, in the reverse direction, Chilean presidents sometimes move into legislative offices after they leave the executive branch: does Frei's experience wielding this authority affect his response to it when he becomes a senator later on down the line? Considering a set of controls for previous experience of the president (legislative branch, executive branch, or none) and previous experience of sitting legislators (or perhaps just the committee or coalition chairs) should help to address this.
\label{sec:org45bd085}
\begin{itemize}
\item \textbf{Actions taken}: Elaborated in the text.
\item \textbf{Explanation}: Frei and Lagos had been legislators prior to assuming presidential office. Bachelet and Piñera had not. After the end of the presidential term only Frei assumed legislative office in the period. With four presidents only, fixed (Model 3) and mixed (Model 4) effects capture prior presidential experience. Appendix Figure A.1 makes clear that, indeed, presidents without legislative experience relied way more on urgencies than those with experience. But the available data cannot answer if this is the explanatory factor behind the surge, or some other factor. We have included a temporal breakdown of urgencies by presidency as part of Table 1 to discuss this propensity in the text.
\end{itemize}
\item Finally, the information in Part B of Table 2 and the related in-text discussion strikes me as a footnote plus appendix type of discussion. It is distracting from the main point and only one category is considered in the analysis, so I don't think including it in the body of the text makes much sense.
\label{sec:org121a342}
\begin{itemize}
\item \textbf{Actions taken}: Conceded, we dropped Part B.
\item \textbf{Explanation}: Former part B now lives in the appendix, where we elaborate on three urgency types. The text now only considers the relevant (``supreme'') urgent bills. Table 1 has a new part B, a breakdown of urgent bills by presidency.
\end{itemize}
\end{enumerate}
\section{Cover letter for RnR}
\label{sec:orgf3fbbd7}
October 10, 2018

Lanny W. Martin
Professor of Political Science
Comparative Politics Editor, Journal of Politics

Department of Social and Political Sciences
Bocconi University
lanny.martin@unibocconi.it

Dear Dr. Buhaug: My co-authors and I have revised our manuscript and are re-submitting it for your consideration and for a second review. 

We addressed all concerns by reviewers, either in the text or in a new on-line appendix. This letter explains how we have done it. 

We accepted all but two points raised by the reviewers---and have corrected or clarified the text and analysis or elaborated based on critiques and recommendations. We quote below the points raised by the reviewers requiring our attention, following each with what we did and where, or did not do and why. 

In addition to reviewers' feedback, we updated all analysis to include 2015 election returns (data that was still unavailable when we prepared the original manuscript) and included secciones that were split in the period of observation due to overpopulation. These secciones had been dropped from the original analysis to save time (recovering them required a good deal of effort). These units are relatively unimportant in sheer numbers (175 overpopulated secciones were split into 5034 new units in the period, out of a total of 66 thousand). But they are concentrated in suburban areas with fast demographic growth since the 1990s. The revised estimates support the the same substantive conclusions, although some individual estimates have changed. 

The new on-line appendix provides detail of our estimation procedure, with a step-by-step explanation of how to prepare data, invoke hypothetical election generation, and specify the Bugs model. At time of publication, we will archive replication code, and data along with this appendix---which will support straightforward replication.

The revised manuscript is 9,768 words long, inclusive (checked with \url{http://app.uio.no/ifi/texcount/online.php}). 
File: redMexBias09.tex
Sum count: 9768
Words in text: 8356
Words in headers: 44
Words outside text (captions, etc.): 1368
Number of headers: 10
Number of floats/tables/figures: 7
Number of math inlines: 89
Number of math displayed: 5

We are confident that the review process has allowed us to improve our manuscript, and hope that the revised version will be acceptable for publication.

Yours sincerely,

Eric Magar, corresponding author
\section{Response to editor accepting to do RnR \textit{<2018-10-10 Wed>}}
\label{sec:orge76c59b}
Dear Lanny, 
It is with great pleasure that I read the good news about our submission. The reviews are constructive, offering substantive advise, and also arrived in due time! I am sure that the review process will improve the manuscript in the hope that it is acceptable for publication in the Journal of Politics. My co-authors and I will gladly proceed with the revise and resubmit. We will send you a revised manuscript within three months. 
Best,
\section{Editor's letter \textit{<2018-10-07 Sun>}}
\label{sec:orgc4fd045}
Ref.:  Ms. No. 182014
Presidents on the Fast Track: Fighting Floor Amendments with Restrictive Rules
The Journal of Politics

Dear Professor Sin,

I have received the reviews of your manuscript, Number 182014, "Presidents on the Fast Track: Fighting Floor Amendments with Restrictive Rules." These are attached at the end of this email. 

The reviews, as you can see, are mixed. While the reviewers are generally sympathetic to what you are trying to do and your general approach to the question, they also raise a variety of important concerns and offer several useful suggestions. Based on the reviews and my own careful reading of the manuscript, I invite you to revise and resubmit your manuscript for further review.

When I receive the revised manuscript I will send it back to Reviewers 2 and 3. Although I do not expect to send it to a wholly new reviewer, I reserve the right to do so, especially if one of the original reviewers is unavailable to read the manuscript a second time.

REVISION: Please address the issues raised by each of your reviewers and make your manuscript revisions accordingly. Along with your revisions to the manuscript, you should prepare an anonymous memorandum addressed to the editors and reviewers. This memo should be included at the top of the electronic file of your revised manuscript. The memo should address the concerns raised by the different reviewers and detail the changes made in the manuscript in response. Information that you do not care to make available to the reviewers should be included in the text "Comments to the Editors" box that is available when you submit the manuscript online.

LENGTH: The revised version should not be any longer than the original submission using the same margins and font size. If you need additional space, you should consider putting nonessential or supplementary materials in an online appendix.

APPENDICES: You are welcome to have an online appendix associated with your paper. The JOP will host online appendices for published articles. Any online appendices must be submitted as a separate file when your revised manuscript is resubmitted online to Editorial Manager. This allows reviewers of the final version of your manuscript to have access to it. Please note that every online-only appendix must be cited at some point in the text.

DATA AND REPLICATION FILES: Note that authors of quantitative papers must submit their data and all associated replication files to the JOP's Dataverse: \url{https://dataverse.harvard.edu/dataverse/jop}. While it is not necessary to provide this information at the R-and-R stage, please note that this is a required step at the "Accept with Revisions" (i.e., "Conditional Accept") stage.

TIMEFRAME: If you decide to undertake revisions, I would prefer to receive the revised manuscript within about three months. If you have not resubmitted the revised manuscript within six months from this letter, you should contact me before doing so to be certain that the invitation to resubmit still stands.

SUBMITTING YOUR REVISION: When you have completed your revisions please log on to Editorial Manager as an author and upload the revised Anonymous Manuscript with the embedded memo to editors and reviewers and, if needed, a separate file with the (anonymous) online appendix. Do not upload an Author Identified version of the manuscript at this point.

Note that while the invitation to revise and resubmit the manuscript does not constitute a commitment to publish, I am aware of the considerable time and effort required to undertake revisions. I would not invite you to do so if I was not confident that you can satisfy the reviewers' concerns and gain a favorable decision. 

I am convinced that this is interesting and important work with the potential to make a significant contribution to the field. I look forward to reading your revised manuscript.

Best regards,
Lanny
\end{document}
